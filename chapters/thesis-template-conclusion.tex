\chapter*{Заключение}
\addcontentsline{toc}{chapter}{Заключение}


В рамках данной работы произведен анализ, разработка и реализация модели микросервисной архитектуры, с паттерном Circuit Breaker, распределённым трассированием и методами нагрузочного тестирования. Основной акцент исследования сделан на измерении задержек возникающих в такой системе, на сравнении готового решения (Istio) с самостоятельно реализованной библиотекой.

Результатами исследования являются:
\begin{itemize}
    \item UML диаграммы, описывающие микросервисную архитектуру кластеров с service mesh и без нее. 
    \item Разработка двух вариантов реализации паттерна Circuit Breaker. Первый вариант основан на использовании инфраструктурного решения (Istio), а второй – на программном подходе с использованием Python-библиотеки. 
    \item Интеграция методов нагрузочного тестирования и распределённого трассирования. Применение инструмента k6 для моделирования нагрузки в сочетании с алгоритмами сбора и анализа трассировочных данных. Количественные оценки производительности микросервисов в различных режимах работы.
    \item Анализ эффективности и времени работы предложенных решений. Проведение сравнительного анализа разработанных подходов по критически важным параметрам позволит сформировать основу для дальнейшей оптимизации и практической реализации микросервисных архитектур. 
\end{itemize}

Предполагаемая область применения результатов исследования охватывает распределённые вычислительные среды, в которых важными характеристиками являются высокая отказоустойчивость и производительность. Практическая значимость работы заключается в возможности проведения обоснованного анализа и выбора оптимального способа реализации паттерна – либо посредством использования service mesh, либо через разработку самописного решения для микросервисов. Данное исследование позволяет оценить рентабельность и эффективность различных методов до реальной эксплуатации системы.

В перспективе планируется расширение исследования за счёт изучения других паттернов и оценки альтернативных решений в области service mesh. Полученные результаты могут стать базой для создания нового поколения высокопроизводительных и отказоустойчивых микросервисных архитектур.

%%% Local Variables:
%%% TeX-engine: xetex
%%% eval: (setq-local TeX-master (concat "../" (seq-find (-cut string-match ".*-3-pz\.tex$" <>) (directory-files ".."))))
%%% End:
