\chapter*{Введение}
\label{sec:afterwords}
\addcontentsline{toc}{chapter}{Введение}


Индустрия разработки программного обеспечения в последние годы активно переходит от монолитных решений к микросервисной архитектуре, обеспечивающей масштабируемость и отказоустойчивость систем. Однако увеличение количества взаимодействующих компонентов существенно повышает вероятность сбоев, что требует внедрения эффективных механизмов обеспечения стабильности. Одним из ключевых решений данной проблемы является паттерн Circuit Breaker, предназначенный для изоляции неисправных сервисов и предотвращения каскадных отказов в распределенных системах.

Актуальность выбранной тематики обусловлена высоким интересом разработчиков облачных платформ, высоконагруженных сервисов и крупных интернет-проектов к надежности и производительности микросервисных решений. Исследования в области реализации паттерна Circuit Breaker важны для повышения устойчивости таких систем и минимизации негативных последствий сбоев.

Концепция паттерна Circuit Breaker была впервые детально описана в 2007 году Майклом Найгардом\cite{Nygard1stEdition}. Значительный вклад в развитие практических решений внесла компания Netflix, выпустив в 2012 году библиотеку Hystrix \cite{hystrix_wiki}. Новый этап развития произошёл в 2017 году с появлением Istio\cite{istio_gluecon2017} – платформы, интегрировавшей функциональность Circuit Breaker на уровне инфраструктуры сервисной сетки. Несмотря на широкое распространение данного подхода, открытым остается вопрос о сравнительной эффективности и влиянии различных реализаций паттерна на производительность микросервисной архитектуры.

Настоящая работа посвящена проведению комплексного сравнительного анализа реализации паттерна Circuit Breaker на основе сервисной сетки Istio и собственной Python-библиотеки, разработанной специально для данного исследования. Оригинальность исследования заключается в прямом сопоставлении двух различных подходов – встроенного инфраструктурного решения и специализированного программного модуля.

Научная значимость данной работы определяется эмпирическим характером анализа, выполненного в кластере K3s с использованием современных инструментов нагрузочного тестирования k6 и системы распределенного трассирования Jaeger. В исследовании делается акцент на сравнение задержек при использовании встроенной в Istio реализации с задержками, возникающими при использовании собственной реализации паттерна на Python, а также на устойчивость к отказам. Полученные результаты имеют практическую ценность, предоставляя разработчикам распределенных систем рекомендации по выбору оптимального подхода для обеспечения отказоустойчивости в микросервисных архитектурах.



В первой главе представлен аналитический обзор современных архитектурных паттернов и технологий, применяемых для обеспечения отказоустойчивости в распределённых микросервисных приложениях. Рассматриваются возможности системы оркестрации, изучаются готовые решения для интеграции паттернов отказоустойчивости. Проводится анализ доступных методов тестирования и средств трассирования. 

Во второй главе предложена модель микросервисной архитектуры, основанная на использовании паттерна Circuit Breaker. Описываются архитектурные особенности каждой из реализаций. Уточняется, что при использовании Istio возникают дополнительные sidecar-контейнеров. Особое внимание уделено процессу передачи трассировок от серверных приложений через data collector в систему распределённого трассирования Jaeger, что обеспечивает детальный анализ задержек.

Третья глава посвящена фактическому анализу архитектуры проекта. Представлены UML-диаграмма развёртывания и диаграмма последовательности, раскрывающие фактическое расположение микросервисов в кластере, способы взаимодействия между частями системы.

Четвёртая глава содержит описание практической реализации программной библиотеки на языке Python. Подробно рассмотрены вопросы настройки и конфигурации сервисной сетки Istio, а также представлен сравнительный анализ производительности, задержек и общей эффективности рассмотренных реализаций.

% \begin{itemize}
% 	\item актуальность:
% 	\begin{itemize}
% 		\item кто и почему в настоящее время интересуется данной проблематикой (в т.ч. для решения каких задач могут быть полезны исслелования в данной области),
% 		\item краткая история вопроса (в формате год-фамилия-что сделал),
% 		\item нерешенные вопросы/проблемы;
% 	\end{itemize}
% 	\item новизна работы (что нового привносится данной работой);
% 	\item оригинальная суть исследования;
% 	\item содержание по главам (по одному абзацу на главу).
% \end{itemize}

% Общий объем введения должен не превышать 1,5 страниц (для ПЗ к УИРам может быть чуть меньше).

%%% Local Variables:
%%% TeX-engine: xetex
%%% eval: (setq-local TeX-master (concat "../" (seq-find (-cut string-match ".*-3-pz\.tex$" <>) (directory-files ".."))))
%%% End:
