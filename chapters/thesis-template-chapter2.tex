\chapter{Моделирование микросервисной архитектуры с применением паттерна Circuit Breaker и алгоритмов распределенного трассирования}

\textbf{Аннотация.} \textit{В  данном разделе представлена теоретическая разработка модели микросервисной архитектуры, использующей паттерн Circuit Breaker для обеспечения устойчивости к сбоям. Детально изучаются модели двух кластеров на базе k3s с реализацией трассирования через промежуточный сборщик и Jaeger. Исследуются подходы к нагрузочному тестированию с использованием k6 и анализу производительности микросервисов в условиях различных режимов работы.}

В данном разделе описываются модели микросервисной архитектуры, ориентированной на повышение устойчивости системы к отказам за счёт применения паттерна Circuit Breaker. Рассматриваются подходы к реализации паттерна в конкретном программном окружении и способы интеграции с существующими сервисами. Также затронуты вопросы организации инфраструктуры на базе k3s с применением инструментов управления конфигурацией (Helm), а также интеграции компонентов мониторинга и трассирования с использованием Fluent Bit и Jaeger. 

  
\section{Модель микросервисной архитектуры с применением паттерна Circuit Breaker}
  
\textbf{Аннотация.} \textit{Подраздел описывает структуру кластерной микросервисной системы, состоящей из серверных компонентов и прокси с реализацией паттерна Circuit Breaker. Рассматриваются два варианта реализации: с использованием Istio Proxy и с применением Python-библиотеки. Описаны принципы взаимодействия компонентов и маршрутизации трафика между узлами внутри k3s кластера.}

  
\section{Интеграция методов нагрузочного тестирования и распределенного трассирования}
  
\textbf{Аннотация.} \textit{В подразделе представлена методология применения инструмента k6 для симуляции нагрузки на микросервисную систему. Описан процесс сбора и анализа трассировочных данных с использованием промежуточного сборщика и Jaeger как системы централизованного трассирования. Рассмотрен поток данных от микросервисов через компоненты трассирования до визуализации в Jaeger UI.}

  
\section{Сравнительный анализ реализаций Circuit Breaker}
  
\textbf{Аннотация.} \textit{Подраздел содержит сравнительный анализ двух подходов к реализации паттерна Circuit Breaker в микросервисной архитектуре: инфраструктурного с использованием Istio и программного на базе Python. Рассмотрены особенности настройки, механизмы определения сбоев и алгоритмы восстановления для обоих вариантов в контексте устойчивости системы.}

  
\section{Выводы}
  
\textbf{Аннотация.} \textit{В подразделе представлены заключительные выводы по результатам теоретической разработки модели микросервисной архитектуры. Обозначены отличия реализации с использование service mesh, от реализации с самостоятельно написанным паттернов. Учитывается эффективность методов распределенного трассирования и нагрузочного тестирования для обеспечения отказоустойчивости системы. Определены ключевые факторы, влияющие на производительность и надежность разработанной архитектуры.}



%%% Local Variables:
%%% TeX-engine: xetex
%%% eval: (setq-local TeX-master (concat "../" (seq-find (-cut string-match ".*-3-pz\.tex$" <>) (directory-files ".."))))
%%% End:
