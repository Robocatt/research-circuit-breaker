 \chapter{Разработка моделей и алгоритмов \dots}

В этой главе описываются разработанные/модифицированные модели/методы/
алгоритмы, или/и описывается применение известных стандартных методов. Также, 
в конце главы обычно приводится общая архитектура программной системы, 
вытекающая из описанной теории. Приведенные ниже заголовки подразделов так же 
весьма примерные и сильно зависят от особенностей конкретной работы.

Формулы и их части необходимо набирать в математическом режиме
(символ \verb|$|). Во избежание переноса длинных формул между строками их 
стоит размещать по центру колонки, например,
\begin{center}
$S a b c = (\lambda x y z. x z (y z)) a b c = a c (b c)$,
\end{center}
\noindent и, если абзац после формулы продолжается, необходимо использовать 
\verb|\noindent|.

Для набора правил вывода можно использовать пакет \texttt{mathpartir.sty}. 
Правила вывода могут быть вынесены в виде рисунка (см. рис. 
\ref{img:inferrules}).

\begin{figure}[t]
  \centering
    \begin{mathpar}
      \inferrule{
        M \to M'
      }{
        N M \to N M'
      } \quad (\mu) \and 
      \inferrule{
        M \to M'
      }{
        M N \to M' N
      } \quad (\nu) \and
      \inferrule{
        M \to M'
      }{
        \lambda x. M \to \lambda x. M'
      } \quad (\xi)
    \end{mathpar}
  \caption{Правила редукции}
  \label{img:inferrules}
\end{figure}

Для оформления определений, теорем, доказательств и т.~п. могут быть 
использованы соответствующие окружения, например:

\begin{definition}
(высказывание)
Высказыванием называется любое истинное или ложное утверждение.
\end{definition}


\section{Модель системы \dots}

\dots




\section{Метод решения задачи для \dots}

\dots





\section{Алгоритмы вычисления \dots}

\dots





\section{Обобщенная архитектура и интерфейсы \dots}

В ряде случаев, все или некоторые результаты проектирования могут быть представлены во второй главе. Обычно же архитектура описывается в третьей главе.

\section{Выводы}

Необходимо перечислить, какие теоретические результаты были получены с 
указанием степени новизны. Например: <<Была разработана такая-то модель. Она 
представляет собой адаптированную версию модели $X$, в которой уравнение $Z$ 
заменено на уравнение $Z'$>>. Еще пример: <<Была предложена такая-то 
архитектура, она отличается от типовой в том-то и том-то. Это позволяет 
избежать таких-то проблем.>>. При этом не следует заниматься <<высасыванием из 
пальца>>: <<Поставленная задача является типовой; для ее решения применены 
стандартные средства (перечислить, какие).>>.

%%% Local Variables:
%%% TeX-engine: xetex
%%% eval: (setq-local TeX-master (concat "../" (seq-find (-cut string-match ".*-3-pz\.tex$" <>) (directory-files ".."))))
%%% End:
