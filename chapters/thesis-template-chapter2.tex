\chapter{Моделирование микросервисной архитектуры с применением паттерна Circuit Breaker и алгоритмов распределенного трассирования}

\textbf{Аннотация.} \textit{ В данном разделе представлена теоретическая разработка модели микросервисной архитектуры, использующей паттерн Circuit Breaker для обеспечения устойчивости к сбоям.  Анализируются два подхода реализации данного паттерна – программная реализация с использованием собственной библиотеке на базе Python и интегрирование паттерна в код и инфраструктурное решение с использованием Istio, конфигурируемое дополнительно, что позволяет выбрать между гибкой настройкой и централизованным управлением отказами. Особое внимание уделяется методам распределённого трассирования с применением промежуточного сборщика данных Fluent Bit и системой трассирования Jaeger, а также нагрузочному тестированию посредством k6, что обеспечивает детальный анализ производительности системы и оперативное обнаружение сбоев. Проведённый анализ выявляет ключевые отличия, влияющие на время работы и задержки, что изучается в последующих главах.}

В данном разделе описываются модели микросервисной архитектуры, ориентированной на повышение устойчивости системы к отказам за счёт применения паттерна Circuit Breaker. Рассматриваются подходы к реализации паттерна в конкретном программном окружении и способы интеграции с существующими сервисами. Также затронуты вопросы организации инфраструктуры на базе k3s с применением инструментов управления конфигурацией (Helm), а также интеграции компонентов мониторинга и трассирования с использованием Fluent Bit и Jaeger. 

  
\section{Модель микросервисной архитектуры с применением паттерна Circuit Breaker}
  
\textbf{Аннотация.} \textit{В подразделе описывается структура кластерной микросервисной системы, состоящей из серверных компонентов и прокси с реализацией паттерна Circuit Breaker. Рассматриваются два варианта реализации архитектуры: с использованием готового решения Istio Proxy и с применением Python-библиотеки. Описаны принципы взаимодействия компонентов и маршрутизации трафика между узлами внутри k3s кластера.}

Рассматриваемая микросервисная архитектура реализована в рамках k3s кластера, где осуществляется маршрутизация трафика между различными узлами посредством внутреннего ClusterIP и TCP соединений. Клиент направляет HTTP-запросы на прокси-компонент, который затем взаимодействует с сервером. Сервер представлен в виде Flask-приложения, обрабатывающего запросы и формирующего ответы, а прокси-компонент выступает в качестве промежуточного звена между клиентом и сервером.

В версии без использования Istio прокси-компонент реализован с применением собственной Python-библиотеки, включающей паттерн Circuit Breaker. Данный паттерн обеспечивает контроль состояния взаимодействия с сервером: при обнаружении ошибок или превышении заданных пороговых значений прокси может немедленно отказать в обслуживании запроса, предотвращая дальнейшие попытки обращения к серверу. При этом веб-приложения сервера и прокси развернуты на отдельных нодах, что повышает отказоустойчивость системы за счёт изоляции компонентов.

При использовании service mesh решения Istio архитектурное устройство сохраняется, однако в поды компонентов прокси и веб-сервера добавляются sidecar-контейнеры, предоставляющие функции проксирования и сетевого взаимодействия. В этом варианте реализация паттерна Circuit Breaker осуществляется средствами Istio, что позволяет исключить наличие данного паттерна в коде прокси-приложения. 

  
\section{Интеграция методов нагрузочного тестирования и распределенного трассирования}
  
\textbf{Аннотация.} \textit{В подразделе представлена методология применения инструмента k6 для симуляции нагрузки на микросервисную систему. Описан процесс сбора и анализа трассировочных данных с использованием промежуточного сборщика и Jaeger как системы централизованного трассирования. Рассмотрен поток данных от микросервисов через компоненты трассирования для последующего визуального анализа в веб-интерфейсе Jaeger.}

В предлагаемой архитектуре интеграция методов нагрузочного тестирования и распределённого трассирования осуществляется посредством специализированных компонентов, развернутых на выделенной k3s ноде. Здесь разворачиваются инструмент k6 для имитации нагрузки, промежуточный сборщик логов fluent bit и система централизованного трассирования Jaeger. Сервер и прокси-компоненты выводят свои логи и трейсы в stdout, после чего fluent bit асинхронно подхватывает, группирует и отправляет их в Jaeger для дальнейшего анализа. Такой подход обеспечивает оперативный сбор данных, позволяя визуализировать распределённые транзакции и выявлять узкие места в работе микросервисной системы.

При использовании Istio принцип работы системы сохраняется: компоненты веб-сервер и прокси, дополнительно снабжённые sidecar-контейнерами, продолжают генерировать логи и трейсы, которые обрабатываются fluent bit и передаются в Jaeger. Нагрузочное тестирование посредством k6 также остаётся неизменным, что позволяет оценить устойчивость системы при различных сценариях нагрузки. Такой унифицированный подход к сбору и анализу данных обеспечивает прозрачность потоков информации и способствует более эффективной диагностике и оптимизации работы микросервисов.
  
\section{Сравнительный анализ реализаций Circuit Breaker}
  
\textbf{Аннотация.} \textit{Подраздел содержит сравнительный анализ двух подходов к реализации паттерна Circuit Breaker в микросервисной архитектуре: инфраструктурного с использованием Istio и программного на базе Python. Описываются отличия полученной архитектуры и реализации. Рассмотрены особенности настройки, механизмы определения сбоев и алгоритмы восстановления для обоих вариантов в контексте устойчивости системы.}

Реализация Circuit Breaker в анализируемых системах осуществляется посредством Helm-чартов, что обеспечивает стандартизированный процесс развертывания и конфигурации. В инфраструктурном подходе с использованием Istio применяется дополнительный YAML-файл – destination rule, в котором описаны параметры Circuit Breaker. В свою очередь, программная реализация на базе Python требует внесения изменений непосредственно в код прокси-компонента, что обуславливает более тесную интеграцию логики обработки сбоев в приложение.

Механизмы определения сбоев в обоих подходах базируются на анализе логов и трассировочных данных. Логи, генерируемые Python кодом, асинхронно собираются Fluent Bit, а трассировочные данные визуализируются в Jaeger UI, что позволяет оперативно выявлять аномалии в работе системы. Такой подход обеспечивает прозрачное наблюдение за потоками данных и помогает быстро диагностировать проблемы, независимо от способа реализации Circuit Breaker.

Быстрое восстановление работоспособности системы достигается за счет возможностей платформы k3s: рестарты контейнеров и перезапуски релизов с помощью Helm позволяют минимизировать время простоя. 

Обе реализации Circuit Breaker демонстрируют высокую отказоустойчивость системы, обеспечивая оперативное обнаружение и восстановление сбоев за счет интеграции в k3s кластер с использованием Helm-чартов. Решение на базе Istio, с применением YAML-конфигурации destination rule, упрощает централизованное управление и мониторинг, исключая необходимость внесения изменений в исходный код приложения. В то время как программная реализация на Python обеспечивает более гибкую настройку логики непосредственно в коде, что может быть предпочтительно для специфичных бизнес-задач.  

\section{Выводы}
  
\textbf{Аннотация.} \textit{В подразделе представлены заключительные выводы по результатам теоретической разработки модели микросервисной архитектуры. Обозначены отличия реализации с использование service mesh, от реализации с самостоятельно написанным паттернов. Учитывается эффективность методов распределенного трассирования и нагрузочного тестирования для обеспечения отказоустойчивости системы. Определены ключевые факторы, влияющие на производительность и надежность разработанной архитектуры.}

В данном разделе сформулированы ключевые выводы по теоретической разработке модели микросервисной архитектуры, основанной на сравнительном анализе двух подходов к реализации паттерна Circuit Breaker. Исследование показало, что применение service mesh, в частности Istio, позволяет централизовать управление сетевыми взаимодействиями за счет использования YAML-конфигураций (destination rule), что упрощает настройку отказоустойчивости и обеспечивает прозрачное внедрение политик обработки сбоев. Альтернативно, собственная реализация на базе Python предусматривает внесение изменений непосредственно в код прокси-компонента, что обеспечивает более гибкую настройку логики Circuit Breaker для специфичных бизнес-сценариев.

Интеграция методов распределённого трассирования и нагрузочного тестирования посредством Jaeger, Fluent Bit и k6 продемонстрировала высокую эффективность в обнаружении аномалий и оперативном восстановлении работоспособности системы. Сбор логов и трейсов, осуществляемый через stdout компонентов и асинхронно обрабатываемый Fluent Bit, позволяет проводить детальный визуальный анализ в Jaeger UI, что существенно ускоряет диагностику и минимизирует время простоя. Автоматизированное восстановление через рестарты контейнеров в k3s и быстрый рестарт релизов посредством Helm-чартов дополнительно повышают устойчивость архитектуры.

Выбор между Istio и собственным решением на Python сводится к компромиссу между удобством централизованного администрирования и гибкостью настройки, а так же временем работы. Дальнейшие практические исследования будут направлены на детальное измерение времени работы и задержек, что позволит оценить влияние выбранного подхода на общую производительность и надежность архитектуры.


%%% Local Variables:
%%% TeX-engine: xetex
%%% eval: (setq-local TeX-master (concat "../" (seq-find (-cut string-match ".*-3-pz\.tex$" <>) (directory-files ".."))))
%%% End:
