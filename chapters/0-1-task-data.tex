\title{Сравнительный анализ реализации микросервисной архитектуры с использованием паттерна Circuit Breaker\\
  на основе K3s и Istio}


\taskdate{28.02.2025}


\projecttasks{%
  % ---------------- Аналитическая часть
  \projecttask{\bfseries\projecttasknum}{\bfseries Аналитическая часть}{}{}{}%
  \projecttask{\projectsubtasknum}%
  {Изучение и сравнительный анализ реализации Kubernetes K3s и сервисной mesh-платформы Istio (преимущества, недостатки, особенности настройки). Изучение паттерна «circuit breaker», логики его работы, особенностей реализации в Istio.}%
  {Текстовый сравнительный анализ систем, схема взаимодействия микросервисов}%
  {1 неделя}{}%

  \projecttask{\projectsubtasknum}%
  {Анализ инструментов для нагрузочного тестирования k6: возможности, интеграция с Kubernetes, изучение возможных типов тестирования. Анализ возможностей системы трассировки Jaeger. Изучение способов интеграции в Kubernetes кластер, изучение процесса формирования метрик и отчётов о задержках.}%
  {Текстовый отчёт, сценарии тестирования, подбор метрик для анализа}%
  {3 неделя}{}%

  \projecttask{\projectsubtasknum}%
  {Оформление расширенного содержания пояснительной записки (РСПЗ)}%
  {Текст РСПЗ}%
  {8 неделя}{}%
   % ---------------- Теоретическая часть
   \projecttask{\bfseries\projecttasknum}{\bfseries Теоретическая часть}{}{}{}%
   \projecttask{\projectsubtasknum}%
   {Создание модели микросервисной архитектуры, в основе которой лежит паттерн «circuit breaker» как метод обеспечения устойчивости системы к сбоям.}%
   {Описанием структуры модели, диаграмма алгоритма работы Circuit Breaker}%
   {5 неделя}{}%
 
   \projecttask{\projectsubtasknum}%
   {Интеграция в модель методов нагрузочного тестирования, добавление алгоритмов распределенного трассирования для сбора и анализа задержек.}%
   {Текстовый отчет со схемой трафика}%
   {6 неделя}{}%
 
   % ---------------- Инженерная часть
   \projecttask{\bfseries\projecttasknum}{\bfseries Инженерная часть}{}{}{}%
   \projecttask{\projectsubtasknum}%
   {Проектирование архитектуры на уровне UML: создание диаграммы компонентов, диаграммы развёртывания для наглядного представления взаимодействий микросервисов, сетевых соединений и конфигурации контейнеров.}%
   {UML диаграммы}%
   {7 неделя}{}%
 
   % ---------------- Технологическая и практическая часть
   \projecttask{\bfseries\projecttasknum}{\bfseries Технологическая и
     практическая часть}{}{}{}%
   \projecttask{\projectsubtasknum}%
   {Разработка и контейнеризация Python-клиента (echo-сервис, отвечающий на входящий запрос), подготовка Docker-образов, загрузка на Docker Hub, описание процедур сборки и развертывания с использованием helm-чартов.}%
   {Исполняемые файлы, исходный текст, Docker-образы, README, yaml-файлы}%
   {8 неделя}{}%
 
   \projecttask{\projectsubtasknum}%
   {Реализация прокси-клиента с паттерном «circuit breaker», создание Docker-образа, конфигурация для приема внешних запросов и перенаправления на echo-сервис. Создание helm-чартов для развертывания в среде k3s.}%
   {Исходный Python код, Docker-образы, README, yaml-файлы}%
   {10 неделя}{}%
 
   \projecttask{\projectsubtasknum}%
   {Подготовка и настройка системы нагрузочного тестирования k6 и трассировки Jaeger на отдельной виртуальной машине, интеграция с приложениями для сбора и анализа метрик, логов и задержек.}%
   {Скрипты k6 и Jaeger, helm-чарты}%
   {11 неделя}{}%
 
   \projecttask{\projectsubtasknum}%
   {Реализация аналогичной схемы (echo и proxy) микросервисов с использованием Istio (установка Istio в кластер, настройка правил «circuit breaker», маршрутизации, сбора метрик и трассировки).}%
   {Манифесты для Istio, конфигурационные файлы}%
   {12 неделя}{}%
 
   \projecttask{\projectsubtasknum}%
   {Проведение нагрузочного тестирования обеих реализаций («чистая» реализация на Python в K3s и на Istio), сбор метрик задержек и пропускной способности с помощью k6 и Jaeger, последующий анализ полученных данных, формирование отчётов и сравнительных графиков.}%
   {Config-файлы тестовых сценариев, графики задержек}%
   {13 неделя}{}%
 
   % ---------------- Оформление ПЗ
   \projecttask{\bfseries\projecttasknum}%
   {\bfseries Оформление пояснительной записки (ПЗ) и иллюстративного материала
     для доклада.}%
   {\bfseries Текст ПЗ, презентация}%
   {\bfseries 13 неделя}{}%
 }
 
 \taskliterature{
  \nocite{Kleppmann2017}
  \nocite{Hightower2017}
  \nocite{Calcote2020}
  \nocite{Richardson2018}
  \nocite{Molyneaux2011}
  \nocite{k6book}
 }
 

% # Подписи

% Для простановки подписи используются слдеющие команды:
% - простая подпись: \sign[<сдвиг>]{<масштаб>}{<FIO>}
% - подпись с датой: \signat[<сдвиг>]{<масштаб>}{<FIO>}{<дата>}
% где
% - <сдвиг> --- необязательный сдвиг подписи по вертикали для правильного
%   расположения относительно строки
% - <масштаб> --- число, используемое для масштарибования изображения подписи
% - <FIO> --- имя файла с подписью, файл должен быть помещен в
%   img/signatures/FIO.png и иметь прозрачный фон
% - <дата> --- дата, которая будет выведена под подписью

% ## Утверждение задания руководителем и студентом
\authortaskapproval{}%
\supervisortaskapproval{}%


% ## Утверждение РСПЗ руководителем, студентом и консультантом
\authorrspzapproval{}%
\supervisorrspzapproval{}%
\consultantrspzapproval{}%

% ## Оценка руководителя за РСПЗ
\supervisorrspzgrade{}%

% ## Утверждение ПЗ руководителем, студентом и консультантом
\authorpzapproval{}%
\supervisorpzapproval{}%
\consultantpzapproval{}%

% ## Оценка руководителя за ПЗ
\supervisorpzgrade{}%

%%% Local Variables:
%%% TeX-engine: xetex
%%% eval: (setq-local TeX-master (concat "../" (seq-find (-cut string-match ".*-1-task\.tex$" <>) (directory-files ".."))))
%%% End:
