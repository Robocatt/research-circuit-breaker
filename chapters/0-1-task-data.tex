\title{Исследование и реализация сферического коня в вакууме\\
  на основе теоретико-множественного подхода}

\taskdate{01.09.2020}

\projecttasks{%
  \projecttask{\bfseries\projecttasknum}{\bfseries Аналитическая часть}{}{}{}%
    % (указываются предмет и цели анализа)
  \projecttask{\projectsubtasknum}%
  {Изучение и сравнительный анализ … с целью…}%
  {Аналитический отчет, список литературы}%
  {}{\signat[4pt]{.12}{VGPetrov}{01.01.2001}}%
  \projecttask{\projectsubtasknum}%
  {Изучение и анализ … для…}%
  {}%
  {}{}%
  \projecttask{\projectsubtasknum}%
  {Анализ … применительно к задачам…}%
  {}%
  {}{}%
  \projecttask{\projectsubtasknum}%
  {Анализ возможностей… (для…, применительно к…, и т.п.)}%
  {}%
  {}{}%
  \projecttask{\projectsubtasknum}%
  {Оформление расширенного содержания пояснительной записки (РСПЗ)}%
  {Текст РСПЗ}%
  {20.10.2020}{} \projecttask{\bfseries\projecttasknum}{\bfseries Теоретическая
    часть}{}{}{}%
    % (указываются используемые и разрабатываемые модели, методы, алгоритмы)
  \projecttask{\projectsubtasknum}%
  {Используется … (модель, метод, алгоритм(ы)…) Модель/ алгоритм/метод...}%
  {}%
  {}{}%
  \projecttask{\projectsubtasknum}%
  {Выбор/разработка…}%
  {}%
  {}{}%
  \projecttask{\projectsubtasknum}%
  {Разработка…}%
  {}%
  {}{}%
  \projecttask{\projectsubtasknum}%
  {Модификация… (алгоритма, модели, и т.п.) для …}%
  {}%
  {}{}%
  \projecttask{\projectsubtasknum}%
  {Адаптация … для…}%
  {}%
  {}{}%
  \projecttask{\bfseries\projecttasknum}{\bfseries Инженерная часть}{}{}{}%
    % (указывается, что конкретно необходимо спроектировать, а также используемые для этого методы, технологии и инструментальные средства)
  \projecttask{\projectsubtasknum}%
  {Проектирование … (системы, подсистемы, модуля…)}%
  {}%
  {}{}%
  \projecttask{\projectsubtasknum}%
  {Использовать методологию проектирования….}%
  {}%
  {}{}%
  \projecttask{\projectsubtasknum}%
  {Разработать архитектуру для… (с учетом требований к…)}%
  {}%
  {}{}%
  \projecttask{\projectsubtasknum}%
  {Результаты проектирования оформить с помощью…. При проектировании
    использовать язык… (например, IDEF, или UML)}%
  {}%
  {}{}%
  \projecttask{\bfseries\projecttasknum}{\bfseries Технологическая и
    практическая часть}{}{}{}%
    % (указывается, что конкретно должно быть реализовано и протестировано, а также используемые для этого методы, инструментальные средства, технологии)
  \projecttask{\projectsubtasknum}%
  {Реализовать… (систему, подсистему, модуль…)}%
  {Исполняемые файлы, исходный текст}%
  {}{}%
  \projecttask{\projectsubtasknum}%
  {Протестировать… с помощью…}%
  {}%
  {}{}%
  \projecttask{\projectsubtasknum}%
  {Разработать тестовые примеры для… }%
  {Исполняемые файлы, исходные тексты тестов и тестовых примеров}%
  {}{}%
  \projecttask{\projectsubtasknum}%
  {Реализация должна иметь форму/обладать качествами...}%
  {}%
  {}{}%
  \projecttask{\projectsubtasknum}%
  {Ожидаемым результатом является программная система/программный
    комплекс/программное обеспечение… со следующими отличительными
    характеристиками…}%
  {}%
  {}{}%
  \projecttask{\projectsubtasknum}%
  {При реализации использовать технологию/платформу…}%
  {}%
  {}{}%
  \projecttask{\bfseries\projecttasknum}%
  {\bfseries Оформление пояснительной записки (ПЗ) и иллюстративного материала
    для доклада.}%
  {\bfseries Текст ПЗ, презентация}%
  {\bfseries 15.12.2021}{} }

\taskliterature{
\nocite{Sychev}
\nocite{Sokolov}
\nocite{Gaidaenko}
}

% # Подписи

% Для простановки подписи используются слдеющие команды:
% - простая подпись: \sign[<сдвиг>]{<масштаб>}{<FIO>}
% - подпись с датой: \signat[<сдвиг>]{<масштаб>}{<FIO>}{<дата>}
% где
% - <сдвиг> --- необязательный сдвиг подписи по вертикали для правильного
%   расположения относительно строки
% - <масштаб> --- число, используемое для масштарибования изображения подписи
% - <FIO> --- имя файла с подписью, файл должен быть помещен в
%   img/signatures/FIO.png и иметь прозрачный фон
% - <дата> --- дата, которая будет выведена под подписью

% ## Утверждение задания руководителем и студентом
\authortaskapproval{\sign[10pt]{.15}{ABIvanov}}%
\supervisortaskapproval{\sign[10pt]{.15}{VGPetrov}}%

% ## Утверждение РСПЗ руководителем, студентом и консультантом
\authorrspzapproval{\sign[10pt]{.15}{ABIvanov}}%
\supervisorrspzapproval{\sign[10pt]{.15}{VGPetrov}}%
\consultantrspzapproval{}%

% ## Оценка руководителя за РСПЗ
\supervisorrspzgrade{10}%

% ## Утверждение ПЗ руководителем, студентом и консультантом
\authorpzapproval{}%
\supervisorpzapproval{\sign[10pt]{.15}{VGPetrov}}%
\consultantpzapproval{}%

% ## Оценка руководителя за ПЗ
\supervisorpzgrade{12}%

%%% Local Variables:
%%% TeX-engine: xetex
%%% eval: (setq-local TeX-master (concat "../" (seq-find (-cut string-match ".*-1-task\.tex$" <>) (directory-files ".."))))
%%% End:
