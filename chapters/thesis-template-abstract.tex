\chapter*{Реферат}
\thispagestyle{plain}
% \pageref{end_of_document}.
Общий объем основного текста, без учета приложений ---
\pageref{end_of_main_text} страниц, с учетом приложений ---
\pageref{end_of_main_text}. Количество использованных источников~--- 22.
Количество приложений~--- 0.

%Ключевые слова: 
\noindent \uppercase{микросервисная архитектура, паттерны отказоустойчивости, контейнеризация, circuit breaker паттерн, service mesh архитектура, Istio}

Целью данной работы является проектирование, внедрение и экспериментальное исследование паттерна Circuit Breaker в микросервисной архитектуре на базе Kubernetes (K3s), что заключается в создании собственной библиотеки для реализации данного паттерна, а также в проведении сравнительного анализа задержек между интегрированным решением Istio и самостоятельно разработанной реализации с целью выявления практических преимуществ и ограничений каждого подхода.

В первой главе проводится анализ современных архитектурных паттернов и технологий, используемых для обеспечения отказоустойчивости распределённых микросервисных приложений.

Во второй главе разрабатывается модель микросервисной архитектуры, основанная на применении паттерна Circuit Breaker, с интеграцией методов нагрузочного тестирования и алгоритмов распределённого трассирования для систематического сбора и последующего анализа задержек.

В третьей главе описывается архитектура проекта. Представляется UML диаграмма развертывания, а так же диаграмма последовательности.

В четвертой главе приводится описание программной реализации библиотеки на языке Python, описывается настройка и конфигурация Istio, приводится детальный сравнительный анализ времени работы, задержек и эффективности каждой из реализаций.

% В приложении \ref{app-format} описаны основные требования к форматированию пояснительных записок к дипломам и (магистерским) диссертациям.

% В приложении \ref{app-structure} представлена общая структура пояснительной записки.

% В приложении \ref{app-manual} приведены некоторые дополнительные комментарии к использованию данного шаблона.

%%% Local Variables:
%%% TeX-engine: xetex
%%% eval: (setq-local TeX-master (concat "../" (seq-find (-cut string-match ".*-3-pz\.tex$" <>) (directory-files ".."))))
%%% End:
