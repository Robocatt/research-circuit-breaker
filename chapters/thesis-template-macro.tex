\usepackage{tabularx}

\usepackage{polyglossia}
\setmainlanguage[numerals=cyrillic]{russian}
\setotherlanguages{english}
\usepackage{csquotes}

\usepackage{xunicode} % some extra unicode support
%\usepackage[utf8x]{inputenc}
\usepackage{xltxtra} % \XeLaTeX macro
\usepackage{fontspec}
\defaultfontfeatures{Ligatures=TeX}

%\setromanfont{Charis SIL}
%\setsansfont{Liberation Sans}
%\setmonofont{PT Mono}
%\setmainfont{Liberation Serif} % this allows to use sans-serif as default font

\usepackage{ifplatform}

\ifwindows
  \newfontfamily{\cyrillicfont}{Times New Roman}
  \setmainfont{Times New Roman}
  \newfontfamily{\cyrillicfonttt}{Courier New}
  \setmonofont{Courier New}
\else
  \setmainfont{Linux Libertine O}
  \setsansfont{Linux Biolinum O}
  \setmonofont[SmallCapsFont={Latin Modern Mono Caps}]{Latin Modern Mono Light}
\fi

%нумерация справа и колонтитулы справа вверху
\usepackage{fancyhdr}
\usepackage[a4paper,left=30mm,right=15mm,top=20mm,bottom=20mm,bindingoffset=0cm]{geometry}%

\usepackage{amsfonts}
\usepackage{amssymb}
\usepackage{amsmath}
\usepackage{amsthm}

\usepackage{calc}
\usepackage{ifthen}
\usepackage{graphicx}
\usepackage{array}
\usepackage{pdfpages}
\usepackage{longtable}
\usepackage{multirow}
\usepackage{indentfirst}
\usepackage[unicode=true]{hyperref}
\usepackage{color}
\usepackage{pgf}
\usepackage{pstheorems}
\usepackage{titling}

% Настройка списков (без лишних вертикальных отступов)
\usepackage{paralist}
\setdefaultenum{1.}{1.}{1.}{1.}
\setdefaultitem{--}{}{}{}
%\setlength\itemsep{-1em}
\let\itemize\compactitem
\let\enditemize\endcompactitem
\let\enumerate\compactenum
\let\endenumerate\endcompactenum
\let\description\compactdesc
\let\enddescription\endcompactdesc
\pltopsep=\smallskipamount
\plitemsep=0pt
\plparsep=0pt
% Команда для отмены разрыва страниц перед списками
\makeatletter 
\newcommand\mynobreakpar{\par\nobreak\@afterheading} 
\makeatother
%%%%%%

\usepackage[singlelinecheck=false,labelsep=endash]{caption}
\captionsetup[table]{justification=justified}
\captionsetup[figure]{justification=justified,name=Рисунок,singlelinecheck=on,font=onehalfspacing}

\usepackage{titlesec}
\titleformat{\chapter}[block]{\centering\normalfont\Large\bfseries}{\thechapter.}{1ex}{}{}
\titlespacing{\chapter}{0pt}{0em}{2em}

\titleformat{\section}[block]{\normalfont\large\bfseries}{\thesection}{1ex}{}{}
\titlespacing{\section}{0pt}{0em}{1ex}

\titleformat{\subsection}[block]{\normalfont\normalsize\bfseries}{\thesubsection}{1ex}{}{}
\titlespacing{\section}{0pt}{0em}{1ex}

	% paragraph и subparagraph -- в тексте, без отступов
\titleformat{\paragraph}[runin]{\normalfont\normalsize\bfseries}{\theparagraph}{0pt}{}{}
\titlespacing{\paragraph}{0pt}{0em}{0ex}

\titleformat{\subparagraph}[runin]{\normalfont\normalsize\bfseries}{\thesubparagraph}{0pt}{}{}
\titlespacing{\subparagraph}{0pt}{0em}{0ex}


% Своё название для Cписка литературы
\usepackage[title, titletoc]{appendix}
\addto\captionsrussian{% Replace "english" with the language you use
	\renewcommand{\contentsname}%
	{Содержание}%
}

%\renewcommand{\appendixname}{Приложение}% Change "chapter name" for Appendix chapters
%\renewcommand{\cftchapdotsep}{\cftdotsep}

\usepackage{mathpartir}

\makeatletter
\let\ps@plain\ps@fancy              % Подчиняем первые страницы каждой главы общим правилам
\makeatother
\pagestyle{fancy}
\fancyhf{}
\fancyfoot[C]{\thepage}
\renewcommand{\headrulewidth}{0pt}
\renewcommand{\footrulewidth}{0pt}
\renewcommand{\baselinestretch}{1.5}
\newcommand{\headertext}[1]{\fancyhead[R]{\tiny{#1}}}

%% Список литературы

\usepackage[
  style=gost-numeric,
  sorting=none,
  language=auto,
  autolang=other
]{biblatex}
\addbibresource{chapters/biblio.bib}

\usepackage{tikz}
\usepackage{hhline}

%\frenchspacing %% изменение расстояние до и после точек в ряде случаев

\renewcommand{\theenumi}{\arabic{enumi}}
\renewcommand{\theenumii}{\arabic{enumii}}
\renewcommand{\theenumiii}{\arabic{enumiii}}
\renewcommand{\theenumiv}{\arabic{enumiv}}

\renewcommand{\labelenumi}{\theenumi.}
\renewcommand{\labelenumii}{\theenumi.\theenumii.}
\renewcommand{\labelenumiii}{\theenumi.\theenumii.\theenumiii.}
\renewcommand{\labelenumiv}{\theenumi.\theenumii.\theenumiii.\theenumiv.}

%\newenvironment{annotation}{\textbf{Аннотация.} \textit}{}
\theoremstyle{plain}
\newtheorem*{annotation}{Аннотация}

\makeatletter
\newcommand*{\projecttypefulldative}[1]{\gdef\@projecttypefulldative{#1}}
\newcommand*{\theprojecttypefulldative}{\@projecttypefulldative}
\newcommand*{\projecttypeshort}[1]{\gdef\@projecttypeshort{#1}}
\newcommand*{\theprojecttypeshort}{\@projecttypeshort}
\newcommand*{\authorfulldative}[1]{\gdef\@authorfulldative{#1}}
\newcommand*{\theauthorfulldative}{\@authorfulldative}
\newcommand*{\authorgroup}[1]{\gdef\@authorgroup{#1}}
\newcommand*{\theauthorgroup}{\@authorgroup}
\newcommand*{\supervisor}[1]{\gdef\@supervisor{#1}}
\newcommand*{\thesupervisor}{\@supervisor}
\newcommand*{\consultant}[1]{\gdef\@consultant{#1}}
\newcommand*{\theconsultant}{\@consultant}
\newcommand{\projecttasks}[1]{\gdef\@projecttasks{#1}}
\newcommand{\theprojecttasks}{\@projecttasks}
\newcommand{\projecttask}[5]{#1 & #2 & #3 & #4 & #5 \\\hline}
\newcommand*{\taskliterature}[1]{\gdef\@taskliterature{#1}}
\newcommand*{\thetaskliterature}{\@taskliterature}
\newcommand*{\taskdate}[1]{\gdef\@taskdate{#1}}
\newcommand*{\thetaskdate}{\@taskdate}
\newcommand*{\supervisortaskapproval}[1]{\gdef\@supervisortaskapproval{#1}}
\newcommand*{\thesupervisortaskapproval}{\@supervisortaskapproval}
\newcommand*{\authortaskapproval}[1]{\gdef\@authortaskapproval{#1}}
\newcommand*{\theauthortaskapproval}{\@authortaskapproval}
\newcommand*{\authorrspzapproval}[1]{\gdef\@authorrspzapproval{#1}}
\newcommand*{\theauthorrspzapproval}{\@authorrspzapproval}
\newcommand*{\supervisorrspzapproval}[1]{\gdef\@supervisorrspzapproval{#1}}
\newcommand*{\thesupervisorrspzapproval}{\@supervisorrspzapproval}
\newcommand*{\consultantrspzapproval}[1]{\gdef\@consultantrspzapproval{#1}}
\newcommand*{\theconsultantrspzapproval}{\@consultantrspzapproval}
\newcommand*{\supervisorrspzgrade}[1]{\gdef\@supervisorrspzgrade{#1}}
\newcommand*{\thesupervisorrspzgrade}{\@supervisorrspzgrade}
\newcommand*{\authorpzapproval}[1]{\gdef\@authorpzapproval{#1}}
\newcommand*{\theauthorpzapproval}{\@authorpzapproval}
\newcommand*{\supervisorpzapproval}[1]{\gdef\@supervisorpzapproval{#1}}
\newcommand*{\thesupervisorpzapproval}{\@supervisorpzapproval}
\newcommand*{\consultantpzapproval}[1]{\gdef\@consultantpzapproval{#1}}
\newcommand*{\theconsultantpzapproval}{\@consultantpzapproval}
\newcommand*{\supervisorpzgrade}[1]{\gdef\@supervisorpzgrade{#1}}
\newcommand*{\thesupervisorpzgrade}{\@supervisorpzgrade}
\makeatother

\newcommand{\sign}[3][0pt]{%
  \tikz[overlay]{\node[yshift=#1]{\includegraphics[scale=#2]{img/signatures/#3.png}}}%
}
\newcommand{\signat}[4][0pt]{%
  \begin{tikzpicture}[overlay]
    \node[xshift=-10pt,yshift=#1](c){\includegraphics[scale=#2]{img/signatures/#3.png}};
    \node[xshift=10pt,yshift=-10pt]{\scriptsize\textit{#4}};
  \end{tikzpicture}
}

\newcommand{\emptyfield}{\tikz[overlay]{\draw[thin,yshift=-1.28ex](0,0)--(5,0)}}

\newcounter{projecttasknumber}
\newcommand{\projecttasknum}{\setcounter{projectsubtasknumber}{0}\stepcounter{projecttasknumber}\theprojecttasknumber.}

\newcounter{projectsubtasknumber}
\newcommand{\projectsubtasknum}{\stepcounter{projectsubtasknumber}\theprojecttasknumber.\theprojectsubtasknumber.}

\usepackage{listings}

\renewcommand{\lstlistingname}{Листинг}

\lstset{
  basicstyle=\linespread{0.94}\ttfamily\small,
  tabsize=2,
  showstringspaces=false,
  columns=flexible,
  numbers=none,
  numberstyle=\tiny\color{gray},
  breaklines=true,
  breakatwhitespace=true,
  framesep=6pt,
  abovecaptionskip=1em,
  captionpos=t,
  extendedchars=true,
  inputencoding=utf8,
  literate={Ö}{{\"O}}1
  {Ä}{{\"A}}1
  {Ü}{{\"U}}1
  {ß}{{\ss}}1
  {ü}{{\"u}}1
  {ä}{{\"a}}1
  {ö}{{\"o}}1
  {~}{{\textasciitilde}}1
  {а}{{\selectfont\char224}}1
  {б}{{\selectfont\char225}}1
  {в}{{\selectfont\char226}}1
  {г}{{\selectfont\char227}}1
  {д}{{\selectfont\char228}}1
  {е}{{\selectfont\char229}}1
  {ё}{{\"e}}1
  {ж}{{\selectfont\char230}}1
  {з}{{\selectfont\char231}}1
  {и}{{\selectfont\char232}}1
  {й}{{\selectfont\char233}}1
  {к}{{\selectfont\char234}}1
  {л}{{\selectfont\char235}}1
  {м}{{\selectfont\char236}}1
  {н}{{\selectfont\char237}}1
  {о}{{\selectfont\char238}}1
  {п}{{\selectfont\char239}}1
  {р}{{\selectfont\char240}}1
  {с}{{\selectfont\char241}}1
  {т}{{\selectfont\char242}}1
  {у}{{\selectfont\char243}}1
  {ф}{{\selectfont\char244}}1
  {х}{{\selectfont\char245}}1
  {ц}{{\selectfont\char246}}1
  {ч}{{\selectfont\char247}}1
  {ш}{{\selectfont\char248}}1
  {щ}{{\selectfont\char249}}1
  {ъ}{{\selectfont\char250}}1
  {ы}{{\selectfont\char251}}1
  {ь}{{\selectfont\char252}}1
  {э}{{\selectfont\char253}}1
  {ю}{{\selectfont\char254}}1
  {я}{{\selectfont\char255}}1
  {А}{{\selectfont\char192}}1
  {Б}{{\selectfont\char193}}1
  {В}{{\selectfont\char194}}1
  {Г}{{\selectfont\char195}}1
  {Д}{{\selectfont\char196}}1
  {Е}{{\selectfont\char197}}1
  {Ё}{{\"E}}1
  {Ж}{{\selectfont\char198}}1
  {З}{{\selectfont\char199}}1
  {И}{{\selectfont\char200}}1
  {Й}{{\selectfont\char201}}1
  {К}{{\selectfont\char202}}1
  {Л}{{\selectfont\char203}}1
  {М}{{\selectfont\char204}}1
  {Н}{{\selectfont\char205}}1
  {О}{{\selectfont\char206}}1
  {П}{{\selectfont\char207}}1
  {Р}{{\selectfont\char208}}1
  {С}{{\selectfont\char209}}1
  {Т}{{\selectfont\char210}}1
  {У}{{\selectfont\char211}}1
  {Ф}{{\selectfont\char212}}1
  {Х}{{\selectfont\char213}}1
  {Ц}{{\selectfont\char214}}1
  {Ч}{{\selectfont\char215}}1
  {Ш}{{\selectfont\char216}}1
  {Щ}{{\selectfont\char217}}1
  {Ъ}{{\selectfont\char218}}1
  {Ы}{{\selectfont\char219}}1
  {Ь}{{\selectfont\char220}}1
  {Э}{{\selectfont\char221}}1
  {Ю}{{\selectfont\char222}}1
  {Я}{{\selectfont\char223}}1
  {…}{\ldots}1
  {–}{-}1
  {\ }{ }1
}

\headertext{}

\authorgroup{Б12-345}
\author{Иванов А. Б.}
\authorfulldative{Иванову Александру Борисовичу}
\supervisor{Петров В. Г.}
\consultant{\emptyfield}
\projecttypefulldative{учебно-исследовательской работе}
\projecttypeshort{УИР}

%%% Local Variables:
%%% TeX-engine: xetex
%%% eval: (setq-local TeX-master (concat "../" (seq-find (-cut string-match ".*-1-task\.tex$" <>) (directory-files ".."))))
%%% End:

\title{Исследование и реализация сферического коня в вакууме\\
  на основе теоретико-множественного подхода}

\taskdate{01.09.2020}

\projecttasks{%
  \projecttask{\bfseries\projecttasknum}{\bfseries Аналитическая часть}{}{}{}%
    % (указываются предмет и цели анализа)
  \projecttask{\projectsubtasknum}%
  {Изучение и сравнительный анализ … с целью…}%
  {Аналитический отчет, список литературы}%
  {}{\signat[4pt]{.12}{VGPetrov}{01.01.2001}}%
  \projecttask{\projectsubtasknum}%
  {Изучение и анализ … для…}%
  {}%
  {}{}%
  \projecttask{\projectsubtasknum}%
  {Анализ … применительно к задачам…}%
  {}%
  {}{}%
  \projecttask{\projectsubtasknum}%
  {Анализ возможностей… (для…, применительно к…, и т.п.)}%
  {}%
  {}{}%
  \projecttask{\projectsubtasknum}%
  {Оформление расширенного содержания пояснительной записки (РСПЗ)}%
  {Текст РСПЗ}%
  {20.10.2020}{} \projecttask{\bfseries\projecttasknum}{\bfseries Теоретическая
    часть}{}{}{}%
    % (указываются используемые и разрабатываемые модели, методы, алгоритмы)
  \projecttask{\projectsubtasknum}%
  {Используется … (модель, метод, алгоритм(ы)…) Модель/ алгоритм/метод...}%
  {}%
  {}{}%
  \projecttask{\projectsubtasknum}%
  {Выбор/разработка…}%
  {}%
  {}{}%
  \projecttask{\projectsubtasknum}%
  {Разработка…}%
  {}%
  {}{}%
  \projecttask{\projectsubtasknum}%
  {Модификация… (алгоритма, модели, и т.п.) для …}%
  {}%
  {}{}%
  \projecttask{\projectsubtasknum}%
  {Адаптация … для…}%
  {}%
  {}{}%
  \projecttask{\bfseries\projecttasknum}{\bfseries Инженерная часть}{}{}{}%
    % (указывается, что конкретно необходимо спроектировать, а также используемые для этого методы, технологии и инструментальные средства)
  \projecttask{\projectsubtasknum}%
  {Проектирование … (системы, подсистемы, модуля…)}%
  {}%
  {}{}%
  \projecttask{\projectsubtasknum}%
  {Использовать методологию проектирования….}%
  {}%
  {}{}%
  \projecttask{\projectsubtasknum}%
  {Разработать архитектуру для… (с учетом требований к…)}%
  {}%
  {}{}%
  \projecttask{\projectsubtasknum}%
  {Результаты проектирования оформить с помощью…. При проектировании
    использовать язык… (например, IDEF, или UML)}%
  {}%
  {}{}%
  \projecttask{\bfseries\projecttasknum}{\bfseries Технологическая и
    практическая часть}{}{}{}%
    % (указывается, что конкретно должно быть реализовано и протестировано, а также используемые для этого методы, инструментальные средства, технологии)
  \projecttask{\projectsubtasknum}%
  {Реализовать… (систему, подсистему, модуль…)}%
  {Исполняемые файлы, исходный текст}%
  {}{}%
  \projecttask{\projectsubtasknum}%
  {Протестировать… с помощью…}%
  {}%
  {}{}%
  \projecttask{\projectsubtasknum}%
  {Разработать тестовые примеры для… }%
  {Исполняемые файлы, исходные тексты тестов и тестовых примеров}%
  {}{}%
  \projecttask{\projectsubtasknum}%
  {Реализация должна иметь форму/обладать качествами...}%
  {}%
  {}{}%
  \projecttask{\projectsubtasknum}%
  {Ожидаемым результатом является программная система/программный
    комплекс/программное обеспечение… со следующими отличительными
    характеристиками…}%
  {}%
  {}{}%
  \projecttask{\projectsubtasknum}%
  {При реализации использовать технологию/платформу…}%
  {}%
  {}{}%
  \projecttask{\bfseries\projecttasknum}%
  {\bfseries Оформление пояснительной записки (ПЗ) и иллюстративного материала
    для доклада.}%
  {\bfseries Текст ПЗ, презентация}%
  {\bfseries 15.12.2021}{} }

\taskliterature{
\nocite{Sychev}
\nocite{Sokolov}
\nocite{Gaidaenko}
}

% # Подписи

% Для простановки подписи используются слдеющие команды:
% - простая подпись: \sign[<сдвиг>]{<масштаб>}{<FIO>}
% - подпись с датой: \signat[<сдвиг>]{<масштаб>}{<FIO>}{<дата>}
% где
% - <сдвиг> --- необязательный сдвиг подписи по вертикали для правильного
%   расположения относительно строки
% - <масштаб> --- число, используемое для масштарибования изображения подписи
% - <FIO> --- имя файла с подписью, файл должен быть помещен в
%   img/signatures/FIO.png и иметь прозрачный фон
% - <дата> --- дата, которая будет выведена под подписью

% ## Утверждение задания руководителем и студентом
\authortaskapproval{\sign[10pt]{.15}{ABIvanov}}%
\supervisortaskapproval{\sign[10pt]{.15}{VGPetrov}}%

% ## Утверждение РСПЗ руководителем, студентом и консультантом
\authorrspzapproval{\sign[10pt]{.15}{ABIvanov}}%
\supervisorrspzapproval{\sign[10pt]{.15}{VGPetrov}}%
\consultantrspzapproval{}%

% ## Оценка руководителя за РСПЗ
\supervisorrspzgrade{10}%

% ## Утверждение ПЗ руководителем, студентом и консультантом
\authorpzapproval{}%
\supervisorpzapproval{\sign[10pt]{.15}{VGPetrov}}%
\consultantpzapproval{}%

% ## Оценка руководителя за ПЗ
\supervisorpzgrade{12}%

%%% Local Variables:
%%% TeX-engine: xetex
%%% eval: (setq-local TeX-master (concat "../" (seq-find (-cut string-match ".*-1-task\.tex$" <>) (directory-files ".."))))
%%% End:


\newcounter{totalfigures}
\newcounter{totaltables}
\newcounter{totallistings}

%%% Local Variables:
%%% TeX-engine: xetex
%%% eval: (setq-local TeX-master (concat "../" (seq-find (-cut string-match ".*-3-pz\.tex$" <>) (directory-files ".."))))
%%% End:
