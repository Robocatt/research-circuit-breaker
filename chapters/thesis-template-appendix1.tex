

\chapter{Приложение A. Листинг программной реализации}\label{app-format}
%\addcontentsline{toc}{chapter}{}

% Istio
Реализация микросервисной архитектуры с Istio.
\lstinputlisting[
    caption = {Python код основного сервера},
    label   = {lst:server-istio.py}    
]{listings/server-istio.py}

\lstinputlisting[
    caption = {Python код прокси сервера},
    label   = {lst:client-istio.py}     
]{listings/client-istio.py}

\lstinputlisting[language=dockerfile,        
                 caption={Полный Dockerfile сборки сервера},
                 label={lst:docker-server-istio}]
                {listings/Dockerfile-istio.txt}
\lstinputlisting[language=dockerfile,        
                 caption={Полный Dockerfile сборки прокси сервера},
                 label={lst:docker-client-istio}]
                {listings/Dockerfile-client-istio.txt}
\lstinputlisting[
    caption = {Описание Helm chart},
    label   = {lst:chart-istio}    
]{listings/helm-istio/Chart.yaml}
\lstinputlisting[
    caption = {Описание переменных в helm},
    label   = {lst:values-istio}    
]{listings/helm-istio/values.yaml}

\lstinputlisting[
    caption = {Описание сервиса серверов},
    label   = {lst:service-server-client-istio}    
]{listings/helm-istio/templates/service-server-client.yaml}
\lstinputlisting[
    caption = {Описание сервиса Jaeger},
    label   = {lst:service-jaeger}    
]{listings/helm-istio/templates/service-jaeger.yaml}
\lstinputlisting[
    caption = {Описание пространства имен},
    label   = {lst:namespace-istio}    
]{listings/helm-istio/templates/namespace.yaml}
\lstinputlisting[
    caption = {Описание FluentBit DaemonSet},
    label   = {lst:fluentbit-daemonset-istio}    
]{listings/helm-istio/templates/fluentbit-daemonset.yaml}
\lstinputlisting[
    caption = {Описание FluentBit конфига},
    label   = {lst:fluentbit-configmap-istio}    
]{listings/helm-istio/templates/fluentbit-configmap.yaml}
\lstinputlisting[
    caption = {Описание Circuit Breaker в Istio},
    label   = {lst:destination-rule-istio}    
]{listings/helm-istio/templates/destination-rule.yaml}
\lstinputlisting[
    caption = {Описание deployment серверов},
    label   = {lst:deployment-server-client-istio}    
]{listings/helm-istio/templates/deployment-server-client.yaml}
\lstinputlisting[
    caption = {Описание deployment Jaeger},
    label   = {lst:deployment-jaeger-istio}    
]{listings/helm-istio/templates/deployment-jaeger.yaml}


\lstinputlisting[
    caption = {Python код экспортера OTLP trace в stdout},
    label   = {lst:otlp_json_conole_exporter}    
]{listings/otlp_json_conole_exporter.py}

% NO istio
Реализация микросервисной архитектуры без Istio во многом похожа. Приведены лишь отличающиеся файлы. 

\lstinputlisting[
    caption = {Python код основного сервера},
    label   = {lst:server-NOistio.py}    
]{listings/server-NOistio.py}

\lstinputlisting[
    caption = {Python код прокси сервера},
    label   = {lst:client-NOistio.py}     
]{listings/client-NOistio.py}

\lstinputlisting[
    caption = {Python код Сircuit Breaker библиотеки},
    label   = {lst:circuit-breaker-lib-core.py}     
]{listings/cb-lib-core.py}

\lstinputlisting[
    caption = {Python код инициализации Сircuit Breaker библиотеки},
    label   = {lst:circuit-breaker-lib-init.py}     
]{listings/cb-lib-init.py}

\lstinputlisting[
    caption = {Конфигурация k6 для нагрузочного тестирования},
    label   = {lst:load-load.yaml}     
]{listings/load-load.yaml}

\lstinputlisting[
    caption = {Конфигурация k6 для тестирования payload},
    label   = {lst:load-size.yaml}     
]{listings/load-size.yaml}

\lstinputlisting[
    caption = {Общий скрипт настройки k6},
    label   = {lst:load-job.yaml}     
]{listings/load-job.yaml}


%%% Local Variables:
%%% TeX-engine: xetex
%%% eval: (setq-local TeX-master (concat "../" (seq-find (-cut string-match ".*-3-pz\.tex$" <>) (directory-files ".."))))
%%% End:
